% Options for packages loaded elsewhere
\PassOptionsToPackage{unicode}{hyperref}
\PassOptionsToPackage{hyphens}{url}
%
\documentclass[
  12pt,
  french,
  oneside]{article}
\usepackage{lmodern}
\usepackage{amssymb,amsmath}
\usepackage{ifxetex,ifluatex}
\ifnum 0\ifxetex 1\fi\ifluatex 1\fi=0 % if pdftex
  \usepackage[T1]{fontenc}
  \usepackage[utf8]{inputenc}
  \usepackage{textcomp} % provide euro and other symbols
\else % if luatex or xetex
  \usepackage{unicode-math}
  \defaultfontfeatures{Scale=MatchLowercase}
  \defaultfontfeatures[\rmfamily]{Ligatures=TeX,Scale=1}
  \setmainfont[]{Lato}
\fi
% Use upquote if available, for straight quotes in verbatim environments
\IfFileExists{upquote.sty}{\usepackage{upquote}}{}
\IfFileExists{microtype.sty}{% use microtype if available
  \usepackage[]{microtype}
  \UseMicrotypeSet[protrusion]{basicmath} % disable protrusion for tt fonts
}{}
\makeatletter
\@ifundefined{KOMAClassName}{% if non-KOMA class
  \IfFileExists{parskip.sty}{%
    \usepackage{parskip}
  }{% else
    \setlength{\parindent}{0pt}
    \setlength{\parskip}{6pt plus 2pt minus 1pt}}
}{% if KOMA class
  \KOMAoptions{parskip=half}}
\makeatother
\usepackage{xcolor}
\IfFileExists{xurl.sty}{\usepackage{xurl}}{} % add URL line breaks if available
\IfFileExists{bookmark.sty}{\usepackage{bookmark}}{\usepackage{hyperref}}
\hypersetup{
  pdftitle={Rapport Modélisation en Epidémiologie},
  pdfauthor={Kevyn Raynal; Clément Violet},
  pdflang={fr-fr},
  pdfkeywords={MEPI, Rennes 1, Projet},
  hidelinks,
  pdfcreator={LaTeX via pandoc}}
\urlstyle{same} % disable monospaced font for URLs
\usepackage[left = 2cm,right = 2cm,top = 2cm,bottom = 2cm]{geometry}
\usepackage{longtable,booktabs}
% Correct order of tables after \paragraph or \subparagraph
\usepackage{etoolbox}
\makeatletter
\patchcmd\longtable{\par}{\if@noskipsec\mbox{}\fi\par}{}{}
\makeatother
% Allow footnotes in longtable head/foot
\IfFileExists{footnotehyper.sty}{\usepackage{footnotehyper}}{\usepackage{footnote}}
\makesavenoteenv{longtable}
\usepackage{graphicx,grffile}
\makeatletter
\def\maxwidth{\ifdim\Gin@nat@width>\linewidth\linewidth\else\Gin@nat@width\fi}
\def\maxheight{\ifdim\Gin@nat@height>\textheight\textheight\else\Gin@nat@height\fi}
\makeatother
% Scale images if necessary, so that they will not overflow the page
% margins by default, and it is still possible to overwrite the defaults
% using explicit options in \includegraphics[width, height, ...]{}
\setkeys{Gin}{width=\maxwidth,height=\maxheight,keepaspectratio}
% Set default figure placement to htbp
\makeatletter
\def\fps@figure{htbp}
\makeatother
\setlength{\emergencystretch}{3em} % prevent overfull lines
\providecommand{\tightlist}{%
  \setlength{\itemsep}{0pt}\setlength{\parskip}{0pt}}
\setcounter{secnumdepth}{-\maxdimen} % remove section numbering
\ifxetex
  % Load polyglossia as late as possible: uses bidi with RTL langages (e.g. Hebrew, Arabic)
  \usepackage{polyglossia}
  \setmainlanguage[]{french}
\else
  \usepackage[shorthands=off,main=french]{babel}
\fi

\title{Rapport Modélisation en Epidémiologie}
\author{Kevyn Raynal \and Clément Violet}
\date{}

\begin{document}
\maketitle

\hypertarget{objectif-du-moduxe8le}{%
\section{Objectif du modèle}\label{objectif-du-moduxe8le}}

L'objectif de ce modèle est d'observer la dynamique de la transmission
de la dengue par le moustique \emph{Aedes albopictus}. Cette maladie
infectieuse menace chaque année près de 40\% de la population mondiale
et infecte chaque année entre 50 et 100 millions de personnes selon
l'OMS. L'originalité de ce modèle est qu'il ne s'intéresse pas à la
principale espèce de moustique vecteur de la dengue qui est \emph{Aedes
aegypti}. Si les auteurs préfèrent s'intéresser à \emph{Aedes
albopictus}, c'est parce que cette espèce a été la cause de plusieurs
épidémies de dengue, cette espèce est plus difficile à contrôler, elle a
un taux de morsure supérieur et est plus compétitive qu'\emph{A.
aegypti}.

Les auteurs ont créé un premier modèle en couplant un modèle classique
SEIR pour modéliser la dynamique de l'infection chez l'Homme avec un
modèle SEI pour modéliser la dynamique de la maladie chez le vecteur.

Le modèle de dynamique chez l'Homme est le suivant :

\begin{equation} \frac{dH_s}{dt} = \lambda H_t - H_s \left(\frac{cV_i}{H_t} + \mu_h\right)\label{eq:eq1}\end{equation}
\begin{equation} \frac{dH_e}{dt} = H_s \frac{cV_i}{H_t} - H_e (\tau_{exh} + \mu_h)\label{eq:eq2}\end{equation}
\begin{equation} \frac{dH_i}{dt} = H_e\tau_{exh} - H_i\left(\tau_{ih} + \alpha + \mu_h\right)\label{eq:eq3}\end{equation}
\begin{equation} \frac{dH_r}{dt} = H_i\left(\tau_{ih}\right) - \mu_h H_r\label{eq:eq4}\end{equation}

Avec les équations \ref{eq:eq1}, \ref{eq:eq2}, \ref{eq:eq3},
\ref{eq:eq4} faisant référence respectivement au nombre de personnes
sensibles, exposées, infectées, immunisées. Le modèle de dynamique
épidémiologique pour le moustique est celui-ci :

\begin{equation} \frac{dV_s}{dt} = \mu_aV_t - V_s \left(\frac{cH_i}{H_t} + mu_a\right)\label{eq:eq5}\end{equation}
\begin{equation} \frac{dV_e}{dt} = V_s \frac{cH_i}{H_t} - V_e \left(\tau_{ex\nu} + \mu_a\right)\label{eq:eq6}\end{equation}
\begin{equation} \frac{dV_i}{dt} = V_e\tau_{ex\nu} - \mu_aV_i \label{eq:eq7}\end{equation}

Les équations \ref{eq:eq5}, \ref{eq:eq6}, \ref{eq:eq7} font quant à
elles référence respectivement aux moustiques sensibles, exposés et
infectés.

Les paramètres de ce modèle sont présentés dans le tableau
\ref{tbl:tab1} adapté de Erickson et al. (2010).

\begin{longtable}[]{@{}lcc@{}}
\caption{Paramètres utilisés pour construire le modèle 1.
\label{tbl:tab1}}\tabularnewline
\toprule
Nom de la variable & Variable & Valeur\tabularnewline
\midrule
\endfirsthead
\toprule
Nom de la variable & Variable & Valeur\tabularnewline
\midrule
\endhead
Taux de croissance de la population humaine & \(\lambda\) &
\(5,8\times 10^{-5}\)\tabularnewline
Taux de mortalité humaine & \(\mu_h\) & \(1/28000\) jours\tabularnewline
Pourcentage journalier de vecteur nécessitant un second repas & \(s_f\)
& \(0.03\)\tabularnewline
Probabilité de mordre un humain & \(b_h\) & \(0,3\)\tabularnewline
Probabilité de transmettre la dengue & \(t_p\) & \(0,4\)\tabularnewline
Probabilité de contact & \(c\) & \(b_h \times t_p\)~\tabularnewline
Inverse du temps d'exposition de l'hôte & \(\tau_{exh}\) & \(1/10\)
jours\tabularnewline
Taux de mortalité des hôtes de la dengue & ~ \(\alpha\) &
\(0,003\)\tabularnewline
Inverse du temps d'infection de l'hôte & \(\tau_{ih}\) & ~ \(1/4\)
jours\tabularnewline
Oeufs par ponte & \(e_p\) & \(30\)\tabularnewline
Inverse du temps de développement des oeufs & \(\tau_e\) &
\(0,361\)\tabularnewline
Taux de mortalité des oeufs & ~\(\mu_e\) & ~\(0,05\)\tabularnewline
Terme de capacité de charge & ~ \(K\) & \(10^{-3}\)\tabularnewline
Inverse du temps de développement des larves & ~ \(\tau_l\) &
\(0,134\)\tabularnewline
Taux de mortalité des larves & \(\mu_l\) & \(0.025\)\tabularnewline
Inverse du temps de développement des nymphes & ~ \(\tau_p\) &
\(0,342\)\tabularnewline
Taux de mortalité des nymphes & \(\mu_p\)~ & \(0,0025\)\tabularnewline
Inverse du temps de développement des adultes immatures & \(\tau_i\) &
\(1\)\tabularnewline
Taux de mortalité des adultes & ~ \(\mu_a\) & \(0,0501\)\tabularnewline
Inverse du temps d'exposition du vecteur & \(\tau_{ex\nu}\) & \(1/9\)
jours\tabularnewline
Temps de nourissage ou de gestation & ~ \(\tau_g\) &
\(0.401\)\tabularnewline
Inverse du temps de reproduction & \(\tau_r\) & \(1\)\tabularnewline
\bottomrule
\end{longtable}

Ce premier modèle relativement simple a été ensuite largement modifié
par les auteurs. Ces derniers ont créé un second modèle en divisant le
cycle de vie du moustique \emph{A. aegypti} en six stades de vie
distincts :

\begin{enumerate}
\def\labelenumi{\arabic{enumi}.}
\tightlist
\item
  Oeufs ;
\item
  Larves ;
\item
  Nymphes ;
\item
  Adultes immatures ;
\item
  Adultes en gestation ;
\item
  Adultes capable de se reproduire.
\end{enumerate}

Seuls les deux derniers stades de ce cycle de vie peuvent donner lieu à
l'infection d'un être humain sensible. En effet, les femelles ayant
atteint ces stades de vies doivent consommer du sang pour pouvoir
assurer la survie de leurs oeufs. Ce second modèle a été évalué par les
auteurs à une température constante de 25°C.

Dans un troisième temps, les auteurs de cet article ont modifié ce
troisième modèle pour influencer certaines variables contrôlant le
modèle 2. Les variables influencées par la température sont :

\begin{itemize}
\tightlist
\item
  Le temps de développement des oeufs ;
\item
  Le temps de développement des larves ;
\item
  Le temps de gestation des oeufs ;
\item
  Le taux de mortalité des adultes.
\end{itemize}

Pour réaliser ces trois modèles, auteurs de cet article se basent sur
quatre hypothèses :

\begin{itemize}
\tightlist
\item
  La population a une exposition homogène aux moustiques ;
\item
  Il n'y a pas de migration humaine ;
\item
  Il n'y a pas de migration de vecteur ;
\item
  Il n'existe qu'un seul sérotype de la dengue.
\end{itemize}

L'hypothèse la plus contraignante pour le modèle est la dernière. Il
existe en réalité plusieurs sérotypes pour ce virus. Or, si une personne
a déjà été infectée par le passé elle devrait se trouver dans le
compartiment immunisé et non pas susceptible. De plus, une personne déjà
infectée par un sérotype peut être infecté par un autre sérotype. Si
cette hypothèse était relaxée, elle complexifierait grandement les
équations concernant les compartiments susceptibles et immunisés.

\hypertarget{reproductibilituxe9-de-larticle}{%
\section{Reproductibilité de
l'article}\label{reproductibilituxe9-de-larticle}}

Pour ce projet, nous avons essayé de reproduire les principaux résultats
de cet article à savoir les principaux graphiques de ce modèle. Le
premier modèle a été codé en R, puisque le nombre d'équations est
relativement restreint.

C'est en essayant de coder par nous-mêmes ce modèle que nous avons
rencontré les points négatifs de cet article : la reproductibilité. Bien
que les équations soient relativement simples dans le premier modèle, il
existe déjà des erreurs de mise en page concernant les équations. Ces
erreurs ont été facilement contournées grâce aux autres modèles de
l'article qui ne sont qu'une extension du premier. Néanmoins, corriger
ces premières erreurs n'a pas été suffisant pour faire fonctionner ce
premier modèle : il y avait d'autres erreurs plus insidieuses.

Le tableau décrivant les valeurs des différents paramètres présenté dans
l'article contient lui aussi des erreurs de mise en page, nous avons
fait le choix de présenter un tableau corrigé (Tab. 1 \ref{tbl:tab1}).
Le premier modèle ne contenait qu'une seule erreur dans les paramètres,
mais la trouver nous a pris plusieurs heures. La correction quant à elle
était simple à mettre en place. Finalement, nous avons réussi à recréer
les résultats pour ce premier modèle (Fig. \ref{fig:model1}).

\begin{figure}
\hypertarget{fig:model1}{%
\centering
\includegraphics{figures/Model1.png}
\caption{Reproduction de la figure 3c de l'article.}\label{fig:model1}
}
\end{figure}

Pour le modèle 2, nous n'avons pas réussi à reproduire les résultats des
auteurs, car nous avons décelé dans les équations pas moins de 6 fautes
différentes. Nous avons essayé de les corriger au mieux, mais nos
résultats ne convergent pas vers ceux trouvés par les auteurs (Fig.
\ref{fig:model2}).

\begin{figure}
\hypertarget{fig:model2}{%
\centering
\includegraphics{figures/Model2.png}
\caption{Résultats du modèle 2}\label{fig:model2}
}
\end{figure}

Il est possible que nos corrections soient également erronées, ou bien
alors qu'il y ait d'autres erreurs dans les paramètres du modèle.
Certains paramètres sont des fonctions dépendantes de la température. Or
l'article ne contient pas ces fonctions mathématiques. L'article
contient uniquement leurs images pour une température de 25°C. Nous
supposons qu'il y a également des erreurs de typographie pour les images
de ces fonctions. Pour les paramètres concernés, nous avons essayé de
nombreuses autres valeurs, mais aucune ne convenait. Les auteurs
indiquent seulement que nous pouvons trouver ces équations dans le
mémoire de master de l'auteur principal. Cependant, aucune version de ce
mémoire n'est disponible en ligne.

Pour le dernier modèle, même s'il nous avait été possible de faire
fonctionner le deuxième modèle, nous n'aurions pas pu le reproduire. En
effet, les auteurs de cet article expliquent avoir fait varier la
température grâce à la température de l'air moyenne de la ville de
Lubbock au Texas. Bien que ces données soient disponibles sur internet,
le fait de ne pas avoir accès aux fonctions mathématiques dont les
images servent de paramètres aux modèles 2 et 3 nous empêche de les
reproduire.

\hypertarget{moduxe9lisation-compluxe9mentaire}{%
\section{Modélisation
complémentaire}\label{moduxe9lisation-compluxe9mentaire}}

Nous avons choisi d'effectuer une analyse de sensibilité car celle-ci
permet de valider le modèle utilisé, d'identifier les paramètres les
plus influents et éventuellement de simplifier le modèle par exemple. La
méthode de Morris (Morris 1991) a été choisie car c'est une méthode qui
ne requiert pas d'hypothèse concernant le modèle et qui est peu coûteuse
en simulations et est une méthode de screening plus représentative que
l'OAT par exemple. En effet, les trajectoires sont choisies
aléatoirement et l'exploration de l'espace est différente et plus
satisfaisante que l'OAT. Nous avons librement adapté le script du site
nanhung.rbind.io (2018) à notre modèle. En effet, après avoir essayé de
développer nous-même l'analyse de sensibilité, celle-ci s'est révélée
infructueuse. Nous avons donc choisi d'opter cette approche qui ne s'est
cependant pas finalisé sur des résultats concluants non plus.
Effectivement, nous n'avons pas réussi à régler les erreurs sorties lors
de l'exécution du script.

\hypertarget{synthuxe8se}{%
\section{Synthèse}\label{synthuxe8se}}

La tentative de reproduire les résultats de cet article scientifique
nous a permis de nous confronter à un problème qui fait grand bruit dans
la communauté scientifique : la crise de la reproductibilité (Baker
2016). Nous avons échoué à pouvoir reproduire l'ensemble des résultats
de cet article.

Ce travail nous a permis de nous rendre compte qu'écrire un article
n'est pas seulement présenter les résultats de ses travaux. Cela va
au-delà en permettant à tout chercheur de reproduire les résultats
obtenus. Il faut donc prendre le plus grand soin à vérifier que toutes
les données nécessaires sont facilement disponibles : équations, code
source, tout devrait être en libre accès. De plus, lors du processus de
publication, chaque chercheur devrait relire avec attention les épreuves
envoyées par l'éditeur de la revue afin de s'assurer qu'aucune erreur de
mise en page n'empêcherait de reproduire les résultats.

\hypertarget{bibliographie}{%
\section*{Bibliographie}\label{bibliographie}}
\addcontentsline{toc}{section}{Bibliographie}

\hypertarget{refs}{}
\leavevmode\hypertarget{ref-Baker_2016}{}%
\textbf{Baker}. 1,500 scientists lift the lid on reproducibility.
\emph{Nature.} Springer Science; Business Media LLC; 533:452‑4.

\leavevmode\hypertarget{ref-Erickson_2010}{}%
\textbf{Erickson et al.} A dengue model with a dynamic aedes albopictus
vector population. \emph{Ecological Modelling.} Elsevier BV;
221:2899‑908.

\leavevmode\hypertarget{ref-Morris_1991}{}%
\textbf{Morris}. Factorial sampling plans for preliminary computational
experiments. \emph{Technometrics.} JSTOR; 33:161.

\leavevmode\hypertarget{ref-nanhung_2018}{}%
(2018). Sensitivity analysis for pbpk model {[}Internet{]}. \emph{Nan
Hung.}

\end{document}
